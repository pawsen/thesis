
\section{Detection}
\label{sec:detection}

This section is only included for completeness, as the signs are well
established. More information can be found in \autocite{worden2000nonlinearity}.

During test, the structure should be excited by a sine-sweep and a mere visual
inspection of the time series will show if nonlinearity is present. Signs
includes skewness of the envelope, discontinuity, jumps and lack of invariance
with increasing force level. The excitation level needs to be at an amplitude
where the nonlinearity is activated.

Random excitation is in general not useful, as the randomness of the amplitude
and phase of the excitation creates ``linearised'' FRFs. At least multiple test
with different forcing levels are required, and still then it might be difficult
to excite the nonlinearities, since the total power of the input spectrum is
spread over the band-limited frequency range used.

The use of impact excitation, as often used in linear analysis, suffers from the
same problems as random excitation. That is, the input is a broad spectrum and
the energy associated with each frequency is low.

Formal methods for detection includes
\begin{itemize}
\item \textit{Homogeneity check} \\
  Comparing the response of two sweeps with different forcing and calculating
  the cross correlation. It is a test of superposition, by testing if the two FRFs
  normalized with forcing overlay as they will for linear systems.
\item \textit{(ordinary) Coherence function} \\
  The coherence function compares power spectral densities (PSDs) and are
  required to be unity for all accessible frequencies for the system to be
  linear {\textit and} free of noise. The advantage is that only one test is needed,
  but the method does not distinguish between cases of noise and
  nonlinearity. Instead it is recommend do use:
\item \textit{Hilbert transforms} \\
  This method detects nonlinearity by doing a Hilbert transform of the FRF,
  which is invariant for a linear FRF.
  A Hilbert transform also only requires one data set and is more sensitive to
  nonlinearity than the coherence function, but still reasonable easy to
  implement. \textcite{kragh2010a} shows that the homogeneity check is superior to
  the Hilbert transforms, having higher sensitivity to nonlinearity.
\item \textit{Wavelet transforms} \\
  Maps a time-history to a time-frequency representation. Fourier transform
  cannot be used, since it is a one-to-one transformation from time to frequency
  domain. This method, as it is also used for characterisation, is included in
  the report and described in section \ref{sec:wavelet_transform}.
\end{itemize}

For all of these methods it is a requirement that the nonlinearities are
activated, eg. the forcing level end frequency interval should be chosen
adequately. Also, the methods are better at detecting nonlinear stiffness than
nonlinear damping. This is due to the fact that the resonance peak is not
shifted as with the stiffness nonlinearity case. Since the FRF is not shifted
but only lowered, the cross correlation coefficient will not decrease as much as
in the stiffness nonlinearity case. In general damping nonlinearity is difficult
to identify.

%%% Local Variables:
%%% mode: latex
%%% TeX-master: "../../report"
%%% End:
