% -*- coding: utf-8 -*-

\chapter{Signal treatment}
\label{cha:signal-treatment}

Signal treatment and excitation type affects how the nonlinearity is seen in the
measured signal.
This chapter treats the types of excitation that are applicable for nonlinear
dynamics.

\section{Types of excitation}
\label{sec:types-excitation}

Different types of excitation gives different response for nonlinear structures,
with certain types being superior for detecting and characterizing and others
smooths out the nonlinearities. For a more in-depth treatment see
\textcite{gatto2010flexible}.


\begin{itemize}
\item \textit{steady state sine excitation}:\\
  Generally {\textit steady state sine excitation} (or {\textit stepped sine}), gives
  the most distinctive nonlinear effects. All the input energy is contained in
  the frequency of excitation. Noise can be removed from the signal by numerical
  filtering and integration, giving a good signal-to-noise ratio (SNR) and thus well
  defined FRFs with clear nonlinear distortions.
  
  The drawback of stepped sine is that it is slow compared to transient and
  random methods. At each step, time is required before the response attains
  steady state condition. It is the preferred method in literature when treating
  nonlinear behavior.

\item \textit{Impact excitation}:\\
  {\textit Impact excitation}, with hammer testing is the standard for measuring
  linear FRFs in industry. Because of the easiness of applying the test {\textit in
    situ} and the broad frequency content in the impulse excites a high number
  of modes.

  For nonlinear testing, the method is not well suited. The energy of each
  individual frequency is small, making it hard to excite structural
  nonlinearities.

\item \textit{random excitation}:\\
  FRFs obtained by band limited {\textit random excitation} will often appears
  linearized due to the randomness of the amplitude and phase of the excitation
  signal.

  \textbf{Lidt forklaring af det!} As for the impact excitation, the energy of
  each frequency is low, making it hard to drive structures into their nonlinear
  regime.

\item \textit{pseudo-random}:\\
  pseudo-random signal is a sum of multiple sine waves, with constant amplitude
  and the phase randomly selected from a uniform distribution between $180
  \degreeC$ and $-180 \degreeC$

  \begin{equation}
    \label{eq:multiple-sine}
    u(t) = \sum_{k=1}^{N_s} A_K \cos \left( 2 \pi k f_0 + \phi_k \right)
  \end{equation}

  where $N_k$ is the number of sine components, $A_k$ and $\phi_k$ is the
  amplitude and random phase of sine component $k$ and $f_0$ is the fundamental
  frequency. Time series of pseudo random signal is obtained by applying the
  inverse DFT to the generated frequency domain representation of the signal.

  To avoid transient effects delay blocks are used, ie. acquisition only starts
  after repeating the same excitation block a number of times.

  The FRF is not distorted by leakage or windowing and due to the continuous
  excitation, the signal has a high SNR.
\end{itemize}



% \section{Signal treatment}
% \label{sec:signal-treatment}

% \subsection{Differentiation}
% \label{sec:differentiation}

% Differentiation is hard without obtaining noisy results. Standard use of
% difference formulas, like the five-point central difference here, will result in noisy signals

% \begin{equation}
%   \label{eq:central_difference}
%   y_i = \frac{1}{12 \Delta t} \left( -y_{i+2} + 8y_{i+1} - 8y_{i-1} + y_{i-2} \right)
% \end{equation}



%%% Local Variables:
%%% mode: latex
%%% TeX-master: "../report"
%%% End:
