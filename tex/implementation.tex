
\chapter{Implementation}
\label{cha:implementation}


The methods presented in the previous chapters are implemented as a library in
python and available online \autocite{paw2017}. The code depends on the standard
python libraries numpy, scipy and matplotlib and can run on all operating
systems. Care have been taking in order to ensure correctness of the
implementations; as much as it have been possible examples from papers and data given
from Liege have been recalculated to ensure that the same results were obtained.
All code is written for this thesis and implement most notably:
\begin{itemize}
\item FEM code: generate FE matrices $\bm M, \bm K$ from a \textit{gmsh}-mesh
  file. Different types of elements are available. Boundary conditions are
  enforced on system matrices.
\item Nonlinear newmark integration with sensitivity analysis
\item Morlet wavelet
\item Restoring force surface
\item Integration, filtering and periodicity calculation of signals
\item NNM continuation and stability
\item HB continuation, stability and bifurcation
\item FNSI, linear and nonlinear parameter estimation. Stabilisation diagram
  for determining model order
\item FRF calculation from periodic signals with multiple periods or from
  nonperiodic signals using spectral densities. Standard deviation of FRF
  between periods is calculated.
\item Modal properties including MAC calculation.
\item Sine sweep(linear and logarithmic) and random periodic excitation.
\item Polynomial and piecewise cubic splines available for identification.
\item Additional types of nonlinearities available for simulation with Newmark,
  HB and NNM: Piecewise linear and hyperbolic tangent(tanh), the latter only for
  damping.
\end{itemize}

As an example of the library approach, the following two examples shows how FNSI
identification and HB continuation are done. Most methods returns a class
storing all information, making multiple runs with different settings easy. To
keep the examples short, they are run with default settings.

\begin{lstlisting}[language=Python,frame=single,breaklines=true,basicstyle=\tiny]
from numpy import np
from vib.signal import Signal
from vib.fnsi import FNSI
from vib.nlforce import NL_force, NL_polynomial
from vib.common import modal_properties

# load time signal
# nsper is the number of signals per period, iu the dof(s) where the force(s) works
mat = np.load(filename + '.npz')
u, ydd, fs, nsper, iu = mat['u'], mat['ydd'], mat['fs'], mat['nsper'], mat['iu']

# create class, integrate and select period(s) (fnsi works on displacements)
signal = Signal(u, fs, ddy=ddy)
signal.get_displ(lowcut, highcut)
signal.cut(nsper, per=[7,8])

# setup nonlinear type and where it works (-1: attached to ground)
inl = np.array([[0,-1], [1,-1]])
enl = np.array([3,3])
nl_pol = NL_polynomial(inl, enl)
nl = NL_force().add(nl_pol)
fmin, fmax = 0, 10
ims, nmodel = 22, 4
fnsi = FNSI(signal, nl, fmin, fmax)
fnsi.calc_EY()
fnsi.svd_comp(ims)
fnsi.id(nmodel)
# Get identified nl parameters and linear frf.
knl, H, He = fnsi.nl_coeff(iu)
# get linear modal parameters
modal = modal_properties(fnsi.A, fnsi.C)
# Done.
\end{lstlisting}
and for HB:

\begin{lstlisting}[language=Python,frame=single,breaklines=true,basicstyle=\tiny]
import numpy as np
from vib.hb.hb import HB
from vib.hb.hbcommon import hb_signal
from vib.nlforce import NL_force, NL_polynomial

# setup system
M = np.array([[m1,0],[0,m2]])
C = np.array([[c1,0],[0,c2]])
K = np.array([[k1+k2, -k2],[-k2, k2+k3]])
inl = np.array([[0,-1], [1,-1]])
enl = np.array([3,3])
knl = np.array([mu1, mu2])
nl_pol = NL_polynomial(inl, enl, knl)
nl = NL_force()
nl.add(nl_pol)
# starting frequency, force amplitude and force dof
f0, famp, fdof = 1, 3, 0

# create HB class, periodic solution and continuation incl stability and
# bifurcation.
hb = HB(M,C,K,nl)
omega, z, stab, lamb = hb.periodic(f0, famp, fdof)
hb.continuation()

# Run for another forcing level
hb2 = HB(M,C,K,nl)
hb2.periodic(f0, 2*famp, fdof)
hb2.continuation(bifurcation=False)
# Done
\end{lstlisting}

All the examples from the previous chapters are included in the examples
directory in the source code, and should run on any python3 installation.

It is noted that the functionality of this library correspond to the Ni2D
software by Nolinsys, a company founded by researches at Liege University, and
should run at the same speed. The python library includes \textit{all} methods
included by Ni2D. The notable difference is that Ni2D have a graphical interface
and requires, besides a license to Ni2D, a license to matlab. The python library
does however have a interactive graphical interface for selecting RFS and
inspecting the evolution of \{stability changes, ie. Floquet exponents, harmonic
components and periodic solution\}(not shown in this report); both are inspired
by Ni2D.


%%% Local Variables:
%%% mode: latex
%%% TeX-master: "../report"
%%% End:
