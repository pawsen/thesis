
\chapter{Numerical considerations}
\label{chap:numerical_appendix}

We want to solve a system of linear equations
\begin{equation}
  \bm A \bm x = \bm b
\end{equation}
where $\bm A$ is not square. How to solve such a system most efficient
computationally wise is not always straight forward and might also depends on
the library being used.


Formally we want to solve
\begin{equation}
  \min_x \norm{\bm A \bm x - \bm b}_2
\end{equation}
such systems are often seen solved as
\begin{equation}
  \bm x = \bm A^+\bm b
\end{equation}
where $\A^+$ is the Moore-Penrose pseudoinverse. This gives $\bm x$ in a least
square sense. But in numerical implementations, one should not use the
Moore-Penrose pseudoinverse, but a iterative least square solver. For instance
one often see \texttt{x = pinv(A)*b;} in matlab, where ideally
\texttt{lsqminnorm(A,b)} should be used.

%%% Local Variables:
%%% mode: latex
%%% TeX-master: "../../report"
%%% End:
