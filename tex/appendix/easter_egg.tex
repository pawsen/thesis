

\chapter{Easter egg}
\label{chap:easter_egg}

Congratulations. You got through and there is no more. This note is not about
vibrations, but some few things on how you(as a student) do linear algebra
fast(er) without any effort.

Whenever you find eigenvalues, do SVD decompistion or solve linear systems, you
use linear algebra. In the end of the seventies a specification for numerical
linear algebra was made called Basic Linear Algebra Subprograms(BLAS), together
with a reference implementation written i \texttt{Fortran 77}. Later came an
extension to BLAS, Linear Algebra Package(LAPACK), also written in \texttt{f77}.
These specifications are the de-facto standard today. So when you call
\texttt{eigs} or \texttt{\textbackslash} in matlab, matlab calls LAPACK and BLAS
compatible routines under the hood.

Now when we run a program, there are many small things that impact performance
we do not think or care about. These things the software takes care of. For
instance if we multiply two matrices, then the program decides how the CPU
should access the memory where the arrays are stored. It decides which rows and
columns of the arrays should be in the fast accessible caches, when memory is
moved, what the block size should be, etc. The optimal settings for these things
differs from architecture to architecture(ie. is it a Pentium, Intel Core, AMD,
etc. cpu).

Thus it makes a big difference if the BLAS implementation is optimised for your
computer. The old reference implementation is not written with these
optimisation in mind, especially because said architectures did not exist back
then. Still it is the standard implementation that follows with many program.

Today (and for the past five years at least) the best implementation of BLAS and
ATLAS references are the intel MKL and Openblas libraries. MLK is free for
students but not open source. Openblas is completely free and open source -
choose that one. To show the difference, figure ... shows the computation time
for matrix multiplication for python compiled with Openblas and the standard
(old reference). The speed difference is remarkable and is the same for all
other linear algebra operations.

Default BLAS \& LAPACK
Dotted two 4096x4096 matrices in 64.22 s.
Dotted two vectors of length 524288 in 0.80 ms.
SVD of a 2048x1024 matrix in 10.31 s.
Cholesky decomposition of a 2048x2048 matrix in 6.74 s.
Eigendecomposition of a 2048x2048 matrix in 53.77 s.

OpenBLAS
Dotted two 4096x4096 matrices in 3.97 s.
Dotted two vectors of length 524288 in 0.74 ms.
SVD of a 2048x1024 matrix in 1.96 s.
Cholesky decomposition of a 2048x2048 matrix in 0.46 s.
Eigendecomposition of a 2048x2048 matrix in 32.95 s.

As an extra example, I wrote a matrix multiplication function in \texttt{C}.
General matrix multiplication is simple, just three loop through rows and
columns and multiply elements (while this kind of looping of course is a sin in
matlab, that is the way you should do it manually in compiled languages like C
and fortran. In fact explicit loops in C are faster than vectorizing in matlab).
The program was 12 times slower than the reference BLAS and ATLAS. The point is,
there is no way you can do linear algebra faster than ``production''
implementations of BLAS and the used version of BLAS matters for speed.

As a closing note: If you want to go really fast, you go parallel using
clusters. To help you with that, use \texttt{PETSc} which does message parsing,
preconditioning and includes a variety of iterative solvers. But \textit{that}
takes a bit of effort.

Ps.
I gave the impression that matlab comes with a slow version of BLAS. That is not
the case. Matlab(closed source and expensive) comes with Intel MKL. But why not
try python, Julia or some of the other open source languages, that are well
suited for scientific calculation and plotting. It is now, when still in
university you have the time.

PPs.
The general matrix multiplication algorithm/implementation scales at
$\mathbb{O}(n^3)$(asymtotic complexity). The fastest matrix multiplication
algorithm changes with size matrix size, since even if they scaly better, the
constant in front might make them unpractical for smaller matrices. A general
matrix algorithm that performs good is the Strassen algorithm, scaling at
$\mathbb{O}(n^{2.81})$.

PPPs.
Performance BLAS libraries are not written in fortran anymore. They are written
in a combination of assembly and C-code.


%%% Local Variables:
%%% mode: latex
%%% TeX-master: "../../report"
%%% End:
