
\chapter{Periodic solution}
\label{chap:per_sol}

\section{Shooting method}
\label{sec:shooting_appendix}

The state space formulated for the full eom \eqref{eq:per_eom} is

\begin{equation}
  \label{eq:app_state_space}
  \dot{\bm z}(t) = \bm L\bm z(t) - \bm g_{nl}(\bm z) + \bm g_{ext}(\omega,t)
\end{equation}
where

\begin{equation}
  \begin{aligned}
    \bm z =
    \begin{bmatrix}
      \bm x \\ \dot{\bm y}
    \end{bmatrix}, \quad
    \bm L =
    \begin{bmatrix}
      \bm 0 & \bm I_n \\
      -\bm M^{-1}\bm K & -\bm M^{-1} \bm C
    \end{bmatrix} \\
    \bm g_{nl} =
    \begin{bmatrix}
      \bm 0 \\ \bm M^{-1} \bm f_{nl}(\bm x, \dot{\bm x})
    \end{bmatrix}, \quad
    \bm g_{ext}
    \begin{bmatrix}
      \bm 0 \\ \bm M^{-1} \bm p_{ext}(\omega, t)
    \end{bmatrix}
  \end{aligned}
\end{equation}

Calculating the periodic motion for the full EOM, instead of the unforced and
undamped case as in section \ref{sec:shooting_method}, only requires changing
the evaluation of the state space to the formula above. The sensitivity analysis
in section \ref{sec:newmark_sens} is derived for the full EOM. Thus at max five
lines of code needs changing.

\section{HB}
\label{sec:hb_appendix}

The derivation of the equations of motion in frequency domain continues from
where it was left in section \ref{sec:harmonic_bal}.

The operators $\bm \nabla$ and $\bm \nabla^2$ used in expressing velocities and
accelerations are given as
\begin{equation}
  \label{eq:hb_nabla}
  \bm \nabla =
  \begin{bmatrix}
    \bm 0 &        &                &        & \\
          & \ddots &                &        & \\
          &        & \bm \nabla_k   &        & \\
          &        &                & \ddots & \\
          &        &                &        & \bm \nabla_{N_H}
  \end{bmatrix}, \quad
  \bm \nabla^2 =
  \begin{bmatrix}
    \bm 0 &        &                &        & \\
          & \ddots &                &        & \\
          &        & \bm \nabla^2_k &        & \\
          &        &                & \ddots & \\
          &        &                &        & \bm \nabla^2_{N_H}
  \end{bmatrix}
\end{equation}
with

\begin{equation}
  \label{eq:hb_nabla2}
  \bm \nabla_k =
  \begin{bmatrix}
    0       & -k\omega \\
    k\omega & 0
  \end{bmatrix}, \quad
  \bm \nabla^2_k =
  \begin{bmatrix}
    -(k\omega)^2 & 0 \\
    0            & -(k\omega)^2
  \end{bmatrix}
\end{equation}

Substitute eqs. \eqref{eq:hb_x_expansion}-\eqref{eq:hb_f_expansion} and
\eqref{eq:hb_vel}-\eqref{eq:hb_acc} into the EOM \eqref{eq:per_eom} to get

\begin{equation}
  \label{eq:hb_eom_subs}
  \bm M((Q(t)\bm\nabla^2) \otimes \mathbb{I}_n)\bm z  +
  \bm C((Q(t)\bm\nabla) \otimes \mathbb{I}_n)\bm z  +
  \bm K((Q(t) \otimes \mathbb{I}_n)\bm z  =
  (\bm Q(t)\otimes I_n) \bm b
\end{equation}

Using the mixed-product product of the Kronecker product $(\bm A \otimes \bm
B)=(\bm C \otimes \bm D) = (\bm A \bm C) \otimes (\bm B \bm D)$ gives


\begin{equation}
  \begin{aligned}
    \bm M((Q(t)\bm\nabla^2) \otimes \mathbb{I}_n) &=
    (1\otimes \bm M) ((\bm Q(t) \bm \nabla^2)\otimes I_n ) =
    (\bm Q(t)\bm \nabla^2)\otimes \bm M \\
    \bm C((Q(t)\bm\nabla) \otimes \mathbb{I}_n) &=
    (1\otimes \bm C) ((\bm Q(t) \bm \nabla)\otimes I_n ) =
    (\bm Q(t)\bm \nabla)\otimes \bm C \\
    \bm K(Q(t) \otimes \mathbb{I}_n) &=
    (1\otimes \bm K) (\bm Q(t) \otimes I_n ) =
    \bm Q(t)\bm \otimes \bm K
  \end{aligned}
\end{equation}

Substituting these into eq. \eqref{eq:hb_eom_subs} gives
\begin{equation}
  ((\bm Q(t)\bm \nabla^2) \otimes \bm M)\bm z +
  ((\bm Q(t)\bm \nabla) \otimes \bm C)\bm z +
  (\bm Q(t) \otimes \bm K)\bm z =
  (\bm Q(t) \otimes I_n) \bm b
  \label{eq:hb_eom_subs2}
\end{equation}

In order to remove the time dependency and to obtain an expression relating the
different Fourier coefficients, a Galerkin procedure projects eq.
\eqref{eq:hb_eom_subs2} on the orthogonal trigonometric basis $\bm Q(t)$

\begin{equation}
  \begin{aligned}
    \left( \left( \frac{2}{T}\int^T_0 \bm Q^T(t) \bm Q(t) \d t \bm \nabla^2
      \right) \otimes \bm M \right) \bm z +
    \left( \left( \frac{2}{T}\int^T_0 \bm Q^T(t) \bm Q(t) \d t \bm \nabla
      \right) \otimes \bm C \right) \bm z \\
    \left( \left( \frac{2}{T}\int^T_0 \bm Q^T(t) \bm Q(t) \d t
      \right) \otimes \bm K \right) \bm z =
    \left( \left( \frac{2}{T}\int^T_0 \bm Q^T(t) \bm Q(t) \d t
      \right) \otimes I_n \right) \bm b
  \end{aligned}
\end{equation}
where $T=2\pi / \omega$ is the period of the external force.

Using that
\begin{equation}
  \frac{2}{T}\int^T_0 \bm Q^T(t) \bm Q(t) \d t = I_{2N_H+1}
\end{equation}
the equations of motion expressed in frequency domain are

\begin{equation}
  %\label{eq:hb_feom}
  (\bm \nabla^2 \otimes \bm M)\bm z + (\bm \nabla \otimes \bm C)\bm z +
  (\mathbb{I}_{2N_H} \otimes \bm K)\bm z =
  (\mathbb{I}_{2N_H} \otimes \mathbb{I}_n )\bm b
\end{equation}
or in more compact form

\begin{equation}
  \label{eq:hb_feom_compact}
  \bm H(\bm z, \omega) = \bm A(\omega) \bm z - \bm b(\bm z) = \bm 0
\end{equation}
where $\bm A$ describes the linear dynamics

\begin{equation}
  \label{eq:hb_A}
  \begin{aligned}
    \bm A &= \bm \nabla^2 \otimes \bm M + \bm \nabla \otimes \bm C +
    \mathbb{I}_{2N_H} \otimes \bm K \\
    &=
    \begin{bmatrix}
      \bm K \\
      & \bm K - \omega^2 \bm M & -\omega \bm C \\
      & \omega \bm C & \bm K - \omega^2 \bm M \\
      & & & \ddots \\
      & & & & \bm K - (N_H \omega)^2 \bm M & -N_H \omega \bm C \\
      & & & & N_H \omega \bm C & \bm K - (N_H \omega)^2 \bm M
    \end{bmatrix}
  \end{aligned}
\end{equation}


\subsection{Stability}
\label{sec:hb_stab_appendix}

\textbf{Skal x skrives som fx $x^*$ for at symbolisere at det er en løsning -
  eller måske ikke med $x^*$, men et tilsvarende symbol?}

To find the stability of a periodic solution, a method known as \textit{Hills
  method} is used to estimate the Floquet exponents.

A periodic solution $\bm x(t)$ satisfying the eom eq. \eqref{eq:per_eom} is
perturbed by an exponential decay
\begin{equation}
  \label{eq:hb_pert}
  \bm p(t) = \bm x(t) + e^{\lambda t}\bm s(t)
\end{equation}
inserting this into the eom eq. \eqref{eq:per_eom},
\begin{equation}
  \bm M\ddot{\bm x} + \bm C\dot{\bm x} + \bm K\bm x +
  (\lambda^2 \bm M \bm s + \lambda(2\bm M \dot{\bm s}) +
  \bm M\ddot{\bm s} + \bm C\dot{\bm s} + \bm K \bm s ) e^{\lambda t} =
  \bm f(\bm p, \dot{\bm p}, \omega, t)
\end{equation}

Following the outline of section \ref{sec:harmonic_bal}, the solution and
perturbation are approximated by Fourier series truncated to $N_H$-th order,
i.e. $\bm x(t) = (\bm Q(t) \otimes I_n)\bm z$ and $\bm s(t) = (\bm Q(t) \otimes
I_n)\bm u$, where $\bm z$ and $\bm u$ contains the Fourier coefficients of $\bm
x$ and $\bm s$, respectively. Following section \ref{sec:hb_appendix}, a
Galerkin is used to obtain

\begin{equation}
  \bm A \bm z +
  (\bm \Delta_2\lambda^2 + \bm \Delta_1 \lambda + \bm A) e^{\lambda t} \bm u =
  \bm b(\bm z + e^{\lambda t} \bm u)
  \label{eq:hb_pert2}
\end{equation}
where $\bm \Delta$ are matrices describing the linear dynamics similar to $\bm
A$ in eq. \eqref{eq:hb_A} and $\lambda$ are Hills coefficients.

\begin{equation}
  \begin{aligned}
    \bm \Delta_1 &= \bm \nabla \otimes 2\bm M + \mathbb{I}_{2N_H+1} \otimes \bm C \\
    &=
    \begin{bmatrix}
      \bm C \\
      & \bm C & -\omega \bm M \\
      & 2\omega \bm M & \bm C \\
      & & & \ddots \\
      & & & & \bm C & -2N_H \omega \bm M \\
      & & & & 2N_H \omega \bm M & \bm C
    \end{bmatrix} \\
    \bm \Delta_2 &= \mathbb{I}_{2N_H+1} \otimes \bm M
  \end{aligned}
\end{equation}

The right-hand side of eq. \eqref{eq:hb_pert2} is evaluated through a Taylor series expansion around
the solution $\bm z$
\begin{equation}
  \bm b(\bm z+e^{\lambda t} \bm u) = \bm b(\bm z) + \frac{\p \bm b}{\p \bm z}\big|_z (e^{\lambda t}\bm u)
\end{equation}

Since $\bm A \bm z - \bm b(\bm z) = \bm 0$ by definition and given that
\begin{equation}
  \bm A - \frac{\p \bm b}{\p \bm z}\big|_z = \bm h_{\bm z}
\end{equation}
eq. \eqref{eq:hb_pert2} is rewritten into the quadratic eigenvalue problem

\begin{equation}
  \left( \bm \Delta_2 \bm \lambda^2 + \bm \Delta_1 \bm \lambda + \bm H_{\bm z}  \right)
  e^{\lambda t} \bm u = \bm 0
\end{equation}


\subsection{NNM}
\label{sec:hb_nnm_appendix}

Following the method above, NNMs can be computed using HB in the following way:

\begin{equation}
  \bm h_{ham} (\bm z, \omega) = (\bm \nabla^2 \otimes \bm M) \bm z +
  (I_{2N_H+1} \otimes \bm K) \bm z + \bm b_{nl} = \bm 0
\end{equation}
where $\bm b_{nl}$ is the vector of the Fourier coefficients of the nonlinear
forces, defined as
\begin{equation}
  \bm f_{nl}(\bm x) = (\bm Q(t)\otimes I_n) \bm b_{nl}
\end{equation}

The phase condition for frequency-domain methods are often to set a Fourier
coefficient of a DOF to zero. Doing this, the equation for a NNM motion is
\begin{equation}
  \label{eq:hb_nnm}
  \bm h_{NNM} =
  \begin{bmatrix}
    \bm h_{ham}(\bm z, \omega) \\ z_i
  \end{bmatrix}
  = \bm 0
\end{equation}
where $z_i$ is a component of $\bm z$.

%%% Local Variables:
%%% mode: latex
%%% TeX-master: "../../report"
%%% End:
