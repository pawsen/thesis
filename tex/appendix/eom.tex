
In this chapter, test data for a simple beam is constructed. The test
data is simulated time series.

To do this, the equations of motion(EOM) for a simple beam is derived together
with nonlinear boundary conditions.
The EOM is converted to an eigenvalue problem(EVP). The EVP is used for a
single mode expansion to get an ordinary differential equation, which gives a
one degree of freedom(DOF) signal used as test data.

\section{Equation of motion}
\label{sec:eom}


\subsection{Derivation of EOM}


The beam model is sketched in figure \ref{fig:eom_model}. It is a fixed slender
beam, simply supported in right side and a nonlinear reaction force $f(U,\dot
U)$ in left side. It is externally excited by harmonic forcing $P(t)$ in
$\tfrac{3}{4}l$ and accelerations are recorded at $\tfrac{1}{2}l$

Gravity is neglected, since it would otherwise cause a non-straight static
equilibrium.

\begin{figure}[!ht]
  \centering
  \resizebox{0.80\textwidth}{!}{\input{fig/eom/problem_sketch.pdf_tex}}
  \caption{Simple model used for validating}
\label{fig:eom_model}
\end{figure}

The beam is assumed to have linear elastic material properties, with flexural
stiffness $EI$, density $\rho$ and damping coefficient $C$, which results in
modal damping, $\zeta$. The beam cross section is rectangular with height $h$
and width $b$.


Considering the infinitesimal deformed beam element in figure
\ref{fig:beam_element}, the equation of motion is derived by applying Newton's
second law. Shear deformations and longitudinal- and rotational inertia are
neglected, as well as gravity as stated above.
For slender beams it is common to assume $V = \O(U^2)$, ie. longitudinal
deformations are second in order as compared to transverse deflections, or more
loosely: axial vibrations are assumed very stiff compared to transverse
vibrations. This justifies the neglection of inertial terms containing $V$
compared to $U$.

Use Newtons second law and taking moment around $(X+dX/2),(U+dU/2)$ to obtain
\begin{align}
  \label{eq:N2_N1}&\sum F_{x} = 0 \Rightarrow N+dN-N=0 \\
  \label{eq:N2_M1}&\sum M = 0 \Rightarrow M+dM-M+TdX-NdU=0 \\
  \label{eq:N2_T1}&\sum F_{y} = m \ddot U \Rightarrow
                    T+dT-T + q dX = \rho A dX \ddot U
\end{align}
where $\tau$ denotes time¸ $q$ is an external force per unit length,
$N=N(X,\tau)$ is the beam elements internal axial force(tension) and
$T=T(X,\tau)$ is the internal transverse force. $U=U(X,\tau)$ is the transverse
deflection and the only deflection considered here.


\begin{figure}[!ht]
  \centering
  \resizebox{0.45\textwidth}{!}{\input{fig/eom/beam_element.pdf_t}}
  \caption{Differential beam element.}
\label{fig:beam_element}
\end{figure}

Dividing eqs. \eqref{eq:N2_N1}-\eqref{eq:N2_M1} by $dX$ gives

\begin{align}
  \label{eq:N2_N2}&N'=0.\\
  \label{eq:N2_T2}&T'= \rho A \ddot U - q \\
  \label{eq:N2_M2}&M'+T-NU'=0.
\end{align}

Eq. \eqref{eq:N2_M2} is differentiated with respect to $X$, which gives
\begin{align}
  \label{eq:N2_M3}M''+T'-N'U'-NU''&=0.
\end{align}

Finally eqs. \eqref{eq:N2_N2} and \eqref{eq:N2_T2} are substituted into
eq. \eqref{eq:N2_M3}

\begin{align}
  \label{eq:N2_M3_2} M'' + \rho A \ddot U - NU'' -q = 0,
\end{align}
which is a partial differential equation for $U$ in space $X$ and time $\tau$.


The normal force is eliminated, since $N' = 0$ and $N(0) = 0$ implies that
$N=0$. Thus only constitutive relations for $M$ is needed

\begin{equation}
  \label{eq:constitute_M}
  M = EI\kappa + CEI\dot\kappa
\end{equation}
where $\kappa = \kappa(X, \tau)$ is the beams curvature at the neutral fiber.

The true nonlinear curvature is given by

\begin{equation}
  \label{eq:kappa_nonlin}
  \kappa = U'' \(1+ \( U'\)^2 \)^{-\tfrac{3}{2}}
\end{equation}
which accounts for finite rotations. Here we only consider a linearized version
of the problem and the rotations are assumed to be small, $(U')^2 \ll 1$, so
that $\kappa \approx U''$. This assumption is valid for small vibrations.


Thus with
\begin{subequations}
  \begin{align}
    M &= EI\kappa + CEI\dot\kappa, \, \kappa \approx U'' \iff \\
    M'' &\approx EIU'''' + CEI\dot U'''' \\
    N &= 0
  \end{align}
\end{subequations}
inserted in eq. \eqref{eq:N2_M3_2} gives

\begin{equation}
  \rho A \ddot U + CEI \dot U'''' + EI U'''' =  q
  \label{eq:eom_dim}
\end{equation}
with boundary conditions

\begin{equation}
  T(0, \tau) = -Q_1, \quad
  M(0, \tau) = 0, \quad
  U(l,\tau) = 0, \quad
  M(l,\tau) = 0
  \label{eq:eom_boundary_cond}
\end{equation}
Where the boundary condition for $M$ is a result of the internal bending
moment being zero at the two supported ends, and $T$ from the reaction force.

Replacing $T$ with eq. \eqref{eq:N2_M2} and $M$ with the constitutive relation
eq. \eqref{eq:constitute_M}, the boundary conditions are written as

\begin{subequations}
\begin{align}
  -EIU'''(0, \tau) + CEI \dot U'''(0, \tau) &= -Q_1 \\
  EIU''(0, \tau) + CEI \dot U''(0, \tau)  &= 0 \\
  U(l, \tau) &= 0 \\
  EIU''(0, \tau) + CEI \dot U''(0, \tau)  &= 0
\end{align}
\label{eq:eom_boundary_cond}
\end{subequations}

\subsection{Nondimensionalized EOM}
\label{sec:nondim_eom}


The variables $X$, $U$ and $\tau$ are non-dimensionalized as

\begin{equation}
  \label{eq:nondim_var}
  X = xl, \qquad U(X,\tau) = u(x,t)l, \qquad \tau = \omega_0^{-1}t,
\end{equation}
where $x$, $u$ and $t$ are non-dimensional position, deflection and time.
$\omega_0$ is a scaling factor with unit Hertz and is chosen as the undamped
natural frequency

\begin{equation}
  \label{eq:omega_def}
  \omega_0 = \sqrt{\frac{EI}{\rho A l^4}}
\end{equation}

Inserting the non-dimensionalized variables \eqref{eq:nondim_var} into the EOM
eq. \eqref{eq:eom_dim} and dividing by $\tfrac{EI}{l^3}$ gives
\begin{equation}
  \ddot u + \dot u'''' + u'''' = \bar q
  \label{eq:eom_nondim}
\end{equation}
where

\begin{align}
  \label{eq:ext_force_nondim}
  \bar q &= \frac{ql^3}{EI} \\
  q(x, t) &= \delta(x - x_0)q_0 \cos(\Omega t)
\end{align}
$q_0$ is the amplitude of the harmonic external excitation with frequency
$\Omega$.


The reaction force $Q_1$ is given by
\begin{equation}
  \label{eq:q1}
  Q_1 = -K_1 U(0, t) - C_1 \dot U(0,t) + \frac{EI}{l^2} f(\bm u(0,t))
\end{equation}
where $f(\bm u(0,t)$ is a non-dimensional, nonlinear force depending on\\
$\bm u(x,t) = \{u(x,t), u'(x,t), \dot u(x,t), \dots \}^T$

The boundary conditions are then non-dimensionalized as
\begin{subequations}
\begin{align}
  -u'''(0, t) &= -\frac{Q_1l^2}{EI} \nonumber \iff \\
  -u'''(0, t) &= k_1u(0,t) + c_1\dot u(0,t) - f(\bm u(0,t))  \\
  u''(0, t) &= 0 \\
  u(1, t) &= 0 \\
  u''(1, t)  &= 0
\end{align}
\label{eq:eom_nondim_boundary_cond}
\end{subequations}

\subsection{Conversion to ODE}
\label{sec:ode}

We want to convert the EOM given by the partial differential equation(PDE) eq.
\eqref{eq:eom_nondim}, to an ordinary differential equation(OEM) which can be
solved numerically. This is done by mode expansion and constructing an
eigenvalue problem(EVP)

\subsubsection{Eigenvalue problem}
\label{sec:evp}

We want to show that the mode shapes are orthogonal, in order to decouple the
EOM. This is done by setting up the EVP.


For the EVP, all nonlinearities must be neglected. Damping is neglected and
boundary loads are assumed linear, $Q_1 = -k_1 u(0, t)$. Free vibrations are
considered.

An time-harmonic solution is assumed
\begin{equation}
u(x,t) = \phi(x) e^{i\omega t}
\label{eq:evp_guess}
\end{equation}
Inserting this into the equation of motion eq. \eqref{eq:eom_nondim} and
dividing by $e^{i\omega t}$ gives the EVP:

The EVP is stated as
\begin{equation}
  \label{eq:evp}
  K \phi = \lambda L \phi,
\end{equation}
where $\phi$ are eigenvectors(mode shapes) and the eigenvalue and operators are
given as

\begin{equation}
  \label{eq:evp_operator}
  \lambda = \omega^2,
  \quad L = 1,
  \quad K = \frac{\d^{4}}{\d x^{4}}\{\}
\end{equation}

The boundary condition for the EVP is found by inserting the solution
\eqref{eq:evp_guess} into \eqref{eq:eom_nondim_boundary_cond} and using the
linear relation for $Q_1$

\begin{equation}
  -\phi'''(0) = k_1\phi(0), \quad
  \phi''(0) = 0, \quad
  \phi(1) = 0, \quad
  \phi''(1)  = 0
\label{eq:evp_boundary_cond}
\end{equation}

Now, to show that the eigenvectors are orthogonal, the EVP must be self-adjoint:
\begin{subequations}
  \label{eq:self_adjont1}
\begin{align}
  &\int_0^1 uLv \d x = \int_0^1 uv \d x \\
  &\int_0^1 uKv \d x = \int_0^1 u''v'' \d x +
    \left[
    uv''' - u''v''
    \right]_0^1
    =
   \int_0^1 u''v'' \d x + k_1u(0)v(0)
\end{align}
\end{subequations}
where $u=u(x)$ and $v=v(x)$ are test functions. The expression for $K$ is
obtained through use of partial integration and the boundary conditions. Since
all remaining terms are symmetric, the operators $L$ and $K$ are self-adjoint
and any eigenvectors are $K$ and $L$ orthogonal \cite{juel2003a}:

\begin{subequations}
\label{eq:orth_relation}
\begin{align}
  &\int_{0}^{1} \phi_{j} L \phi_{k} \d x = \int_{0}^{1} \phi_{j} \phi_{k} \d x = 0  \label{eq:orth_relL} \\
  &\int_{0}^{1} \phi_{j} K \phi_{k} \d x = \int_{0}^{1} \phi_{j} \phi_k'''' =
  \int_{0}^{1} \phi''_{j} \phi''_{k} \d x + k_1\phi_j(0) \phi_k(0) = 0 \label{eq:orth_relK} \\
  &\int_{0}^{1} \phi_{k} \delta(x-x_0) \d x = \phi_k(x_0)  \label{eq:orth_reldelta}
\end{align}
\end{subequations}
for any $j \neq k$. The value when $j=k$ depends on the given mode shape. The
last relation is a property of the Dirac delta function.


\subsubsection{Mode expansion}
\label{sec:mode_expansion}


We assume that the displacement field $u$ is sufficiently smooth and can be
approximated by the $n$-mode expansion
\begin{equation}
  \label{eq:n-mode-expansion}
  u(x,t) = \sum_{j}^n q_j(t) \phi_j(x)
\end{equation}
where $q_j$ is the mode partitioning factor and $\phi_j$ is chosen to be the
$j$-linear undamped mode shape as given by the EVP.


The eom \eqref{eq:eom_nondim} is multiplied by $\phi_k$ and integrated over
$x=0..1$:

\begin{equation}
  \label{eq:exp1}
  \begin{split}
  \int_0^1 \phi_k(x) \ddot u(x,t) \d x +
  \int_0^1 \phi_k(x) \dot u''''(x,t) \d x +
  \int_0^1 \phi_k(x) u''''(x,t) \d x \\
  = q_0 \cos(\Omega t) \int_0^1 \phi_k(x) \delta\(x - x_0\) \d x
\end{split}
\end{equation}
By use of partial integration, \eqref{eq:exp1} is rewritten as

\begin{equation}
  \label{eq:exp2}
  \begin{split}
    \int_0^1 \phi_k(x) \ddot u(x,t) \d x +
    \int_0^1 \phi_k''(x) u''(x,t) \d x \\
    = q_0 \cos(\Omega t) \int_0^1 \phi_k(x) \delta\(x - x_0\) \d x +
    \left[
      \phi_k(x)'u''(x,t) - \phi_k(x)u'''(x,t)
    \right]_0^1
\end{split}
\end{equation}
where the limits on the right hand side is replaced by the boundary conditions
\eqref{eq:eom_nondim_boundary_cond}. Insert the modal expansion
\eqref{eq:n-mode-expansion} to get

\begin{equation}
  \label{eq:exp3}
  \begin{split}
    \sum_{j=1}^n \left(
      \ddot q_j(t) \int_0^1 \phi_j(x) \phi_k(x) \d x +
      q_j(t) \int_0^1 \phi_j''(x)\phi_k''(x) \d x
     \right)\\
     = q_0 \cos(\Omega t) \int_0^1 \phi_k(x) \delta\(x - x_0\) \d x +
     \phi_k(0)\left[ k_1u(0,t) + c_1 \dot u(0,t) - f(\bm u(0,t))  \right]
\end{split}
\end{equation}


The boundary stiffness is rewritten using the mode expansion
\eqref{eq:n-mode-expansion} and the $K$ orthogonality relation
\eqref{eq:orth_relK}.

\begin{equation}
  \label{eq:bound_stiffness}
  k_1u(0,t) \phi _k(0) = \sum_{j=1}^n q_j(t) k_1 \phi_j(0) \phi_k(0) =
  q_k(t) k_1 \phi_k^2(0) - \sum_{j=1,j\neq k}^n q_j(t) \int_0^1 \phi_j'' \phi_k'' \d x
\end{equation}

Insert the $L$ orthogonality relation \eqref{eq:orth_relL} for $j\neq k$ and the
new expression for the boundary stiffness \eqref{eq:bound_stiffness} into
\eqref{eq:exp3}

\begin{equation}
  \label{eq:exp4}
  \begin{split}
    q_k(t) \( \int_0^1 \phi_k''(x)^2 \d x + k_1 \phi_k(0)^2 \)
    + \ddot q_k(t) \int_0^1 \phi_k(x)^2 \d x \\
     = q_0 \cos(\Omega t) \phi_k(x_0) +
     \phi_k(0)\left[ c_1 \dot u(0,t) - f(\bm u(0,t))  \right]
\end{split}
\end{equation}

Move boundary damping to the left hand side and use the Reyleigh qoutient
definition of $\omega_j$:

\begin{equation}
  \label{eq:exp5}
  \begin{split}
    \(\ddot q_k(t) + \omega_k^2 q_k^2 \)
    \int_0^1 \phi_k''(x)^2 \d x +
    \sum_{j=1}^n \dot q_j c_1 \phi_j(0) \phi_k(0)
     = q_0 \cos(\Omega t) \phi_k(x_0) - \phi_k(0) f(\bm u(0,t))
\end{split}
\end{equation}

Divide by $\int_0^1 \phi_k(x)^2 \d x$ to get a classic mass-stiffness-damper
system

\begin{equation}
  \label{eq:mass_stiff_damp_sys}
  M_{jk} \ddot q_j(t) + C_{jk} \dot q_k(t) + K_{jk} q_k = F_k(\bm q)
\end{equation}

where the linear system matrices are given by

\begin{subequations}
  \label{eq:sys_matrix}
\begin{align}
  M_{jk} &= \delta_{jk} \\
  C_{jk} &= \frac{c_1\phi_j(0) \phi_k(0)}{\int_0^1 \phi_k(x)^2\d x}\\
  M_{jk} &= \delta_{jk} \omega_k
\end{align}
\end{subequations}

and the nonlinear boundary force as

\begin{equation}
  \label{eq:bound_force}
  F_j(\bm q(t)) = \phi_k(0) f(\bm u(0,t))
\end{equation}

\FloatBarrier

%%% Local Variables:
%%% mode: latex
%%% TeX-master: "../report"
%%% End:
