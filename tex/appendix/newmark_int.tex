
\chapter{Newmark integration}
\label{chap:newmark-integration}

Newmarks time integration is used for solving the initial value problem defined by
the EOM,

\begin{align}
  \label{eq:nm_eom}
    \begin{aligned}
  &\bm M \ddot{\bm x}(t) + \bm C_v \dot{\bm x}(t) + \bm K \bm x(t) +
    \bm f_{nl} \left( \bm x(t), \dot{ \bm x}(t) \right) = \bm p(t) \\
  &\qquad \bm z_0 =
  \begin{bmatrix}
    \bm x(0) \\
    \dot{\bm x}(0)
  \end{bmatrix}
  =
  \begin{bmatrix}
    \bm x_0 \\
    \dot{\bm x}_0
  \end{bmatrix}
\end{aligned}
\end{align}
where $\bm z_0$ are the initial conditions. $\bm z$ is the state vector.

The associated sensitivity problem, ie. how sensitive the current motion is to
the initial conditions, is given by
\begin{equation}
  \label{eq:nm_sens}
  \begin{aligned}
  &\bm M  \frac{\p \ddot{ \bm x}(t)}{\p \bm z_0}  +
  \bm C_v \frac{\p \dot{\bm x}(t)}{\p \bm z_0}  +
  \bm K \frac{\p \bm x(t)}{\p \bm z_0}  +
    \frac{\p \bm f_{nl}}{\p \bm x} \bigg|_{\bm x(t)}
    \frac{\p \bm x(t)}{\p \bm z_0} +
    \frac{\p \bm f_{nl}}{\p \dot{\bm x}} \bigg|_{\dot {\bm x}(t)}
    \frac{\p \dot{ \bm x}(t)}{\p \bm z_0} =
  \bm 0 \\
  &\qquad
  \begin{bmatrix}
    \frac{\p \bm x(0)}{\p \bm z_0} \\
    \frac{\p \dot{\bm x}(0)}{\p \bm z_0}
  \end{bmatrix}
  = \bm I
    \end{aligned}
\end{equation}
where $\frac{\p \ddot{ \bm x}(t)}{\p \bm z_0}$ is implied to mean
$\frac{\d^2}{\d t^2} \left[\frac{\p \bm x(t)}{\p \bm z_0} \right]$

\section{Solving EOM}
\label{sec:nm_solve_eom}


The solution of the EOM is done by an implicit nonlinear Newmark scheme.
Description of the method can be found in \textcite{krenk2009non}.

Here follows a more throughout derivation. The algorithm itself, for fixed time
step, is shown in pseudo-code in Algorithm \ref{algo:nonlin_newmark}.


The residual of the EOM eq. \eqref{eq:nm_eom} is

\begin{equation}
  \label{eq:nm_res}
  \bm r (\bm x) = \bm M \ddot{\bm x}(t) + \bm C \dot{\bm x}(t) + \bm K \bm x(t) +
    \bm f_{nl} \left( \bm x(t), \dot{ \bm x}(t) \right) - \bm p(t) = 0
\end{equation}

From time $t$ to $t+h$, a Taylor series expansion of the velocity and
displacement with respect to $h$ gives

\begin{equation}
  \label{eq:nm_taylor}
  \begin{aligned}
    \dot {\bm x}_{t+h} &= \dot{\bm x}_t + (1-\gamma)h \ddot{\bm x}_t + \gamma h \ddot{\bm x}_{t+h} \\
    {\bm x}_{t+h} &= \bm x_t + \dot{\bm x}_t + (\frac{1}{2}-\beta)h^2 \ddot{\bm x}_t + \beta h^2 \ddot{\bm x}_{t+h}
  \end{aligned}
\end{equation}
where $\gamma$ and $\beta$ are weighting constants. Often $\gamma =\frac{1}{2},
\beta = \frac{1}{4}$ is chosen, which is the Average Acceleration method. This
gives a scheme that is unconditionally stable and converges as
$\mathcal{O}(h^2)$

This can be rewritten as

\begin{align}
  \label{eq:nm_taylor2}
  &\begin{aligned}
    \dot {\bm x}_{t+h} &= \dot{\bm x}^*_{t+h} + \gamma h \ddot{\bm x}_{t+h} \\
    \bm x_{t+h} &= \bm x^*_{t+h} + \beta h^2 \ddot{\bm x}_{t+h}
  \end{aligned} \\
  &\begin{aligned}
    \dot {\bm x}^*_{t+h} &= \dot{\bm x}_t + (1-\gamma)h \ddot{\bm x}_t  \\
    {\bm x}^*_{t+h} &= \bm x_t + \dot{\bm x}_t + (\frac{1}{2}-\beta)h^2 \ddot{\bm x}_t
  \end{aligned}
\end{align}
where $(^*)$ denotes prediction. The prediction depends on the previous time
step $t$ and implies the prediction $\ddot{\bm x}^*_{t+h} = 0$.

Rewriting eq. \eqref{eq:nm_taylor2} again
\begin{equation}
  \label{eq:nm_scheme}
  \begin{aligned}
    &\ddot{\bm x}_{t+h} = \frac{1}{\beta h^2} (\bm x_{t+h} - \bm x^*_{t+h}) \\
    &\dot{\bm x}_{t+h} = \dot{\bm x}^*_{t+h} \frac{\gamma}{\beta h} (\bm x_{t+h} - \bm x^*_{t+h})
  \end{aligned}
\end{equation}

Substituting this into the residual eq. \eqref{eq:nm_res}, the equation is only
expressed in terms of $\bm x_{t+h}$. Due to the nonlinear stiffness and
dissipation, the residual equation have to be solved iteratively by Newton-Raphson
iterations.


\begin{equation}
  \label{eq:nm_nr}
  \bm r(\bm x^k_{t+h}) + \bm S(\bm x^k_{t+h}) \bm \Delta \bm x = 0
\end{equation}

The iteration matrix $\bm S$, also called the tangent stiffness, is given by

\begin{equation}
  \label{eq:nm_tangent}
  \bm S(\bm x^k_{t+h}) = \left[ \frac{\d \bm r}{\d \bm x} \right]_{\bm x^k_{t+h}} =
  \frac{1}{\beta h^2} \bm M + \frac{\gamma}{\beta h} \bm C + \bm K +
  \frac{\p \bm f_{nl}}{\p \bm x} +
  \frac{\gamma}{\beta h} \frac{\p \bm f_{nl}}{\p \dot{ \bm x}}
\end{equation}

Thus the correction is found by solving eq. \eqref{eq:nm_nr}

\begin{equation}
  \label{eq:nm_nr2}
  \Delta \bm x^k_{t+h} = - \bm S(\bm x^k_{t+h})^{-1} \bm r(\bm x^k_{t+h})
\end{equation}
and updating the state variables

\begin{equation}
  \begin{aligned}
    &\bm x^{k+1}_{t+h} = \bm x^k_{t+h} + \Delta \bm x^k_{t+h} \\
    &\dot{\bm x}^{k+1}_{t+h} = \dot{\bm x}^k_{t+h} +
    \frac{\gamma}{\beta h} \Delta \bm x^k_{t+h} \\
    &\ddot{\bm x}^{k+1}_{t+h} = \ddot{\bm x}^k_{t+h} +
    \frac{1}{\beta h^2} \Delta \bm x^k_{t+h} \\
  \end{aligned}
\end{equation}

The Newton iterations are carried out until some norm of residual value is below
a given tolerance. \textcite{cook2007concepts} have different examples of norms that
can be used.


\section{Sensitivity Equations}
\label{sec:newmark_sens}


At the end of each time step, the sensitivity matrix $\left[ \frac{\p \bm x}{\p
    \bm z_0} \right]$ is found by solving eq. \eqref{eq:nm_sens}.
Using Newmarks scheme given by eq. \eqref{eq:nm_scheme}, at time step $t+h$ one have

\begin{align}
\label{eq:nm_sens_xdd}
  &\left[ \frac{\p \ddot{\bm x}}{\p \bm z_0} \right]_{t+h} =
    \frac{1}{\beta h^2} \left(
    \left[ \frac{\p \bm x}{\p \bm z_0} \right]_{t+h} -
    \left[ \frac{\p \bm x}{\p \bm z_0} \right]^*_{t+h}
    \right) \\
\label{eq:nm_sens_xd}
  &\left[ \frac{\p \dot{\bm x}}{\p \bm z_0} \right]_{t+h} =
    \left[ \frac{\p \dot{\bm x}}{\p \bm z_0} \right]^*_{t+h} +
    \frac{\gamma}{\beta h^2} \left(
    \left[ \frac{\p \bm x}{\p \bm z_0} \right]_{t+h} -
    \left[ \frac{\p \bm x}{\p \bm z_0} \right]^*_{t+h}
    \right)
\end{align}
the predictions are given as (* denote prediction)

\begin{align}
  \label{eq:nm_sens_predict}
  &\left[ \frac{\p \dot{\bm x}}{\p \bm z_0} \right]^*_{t+h} =
    \left[ \frac{\p \dot{\bm x}}{\p \bm z_0} \right]_{t} +
    (1-\gamma)h \left[ \frac{\p \ddot {\bm x}}{\p \bm z_0} \right]_{t} \\
  &\left[ \frac{\p \bm x}{\p \bm z_0} \right]^*_{t+h} =
    \left[ \frac{\p \bm x}{\p \bm z_0} \right]^*_{t}
    h \left[ \frac{\p \dot {\bm x}}{\p \bm z_0} \right]_{t} +
    h^2(\frac{1}{2} - \beta) \left[ \frac{\p \ddot{\bm x}}{\p \bm z_0} \right]_{t}
\end{align}


By rearranging and substitution eqs. \eqref{eq:nm_sens_xdd} and
\eqref{eq:nm_sens_xd} into the linear sensitivity eq. \eqref{eq:nm_sens}, the
sensitivity acceleration matrix at time $t+h$ is found as

\begin{align}
  &\frac{1}{\beta h^2}
  \left[
    \frac{1}{\beta h^2} \bm M +
    \frac{\gamma}{\beta h} \bm C +
    \bm K +
    \frac{\p \bm f_{nl}}{\p \bm x} +
    \frac{\gamma}{\beta h} \frac{\p \bm f_{nl}}{\p \dot{\bm x}}
  \right]
    \left[ \frac{\p \ddot{\bm x}}{\p \bm z_0} \right]_{t+h} = \nonumber \\
 & -
  \left(\bm K + \frac{\bm \p f_{nl}}{\p \bm x} \right)
  \left[ \frac{\p \bm x}{\p \bm z_0} \right]^*_{t+h}
  -
  \left(\bm C + \frac{\bm \p f_{nl}}{\p \dot {\bm x}} \right)
  \left[ \frac{\p \dot{\bm x}}{\p \bm z_0} \right]^*_{t+h}
  \label{eq:nm_sens_sol}
\end{align}
from where the sensitivity matrix is found by \eqref{eq:nm_sens_xdd}.


Thus by marching in time, the current motion and its sensitivity with respect to
initial conditions are found.

\section{Algorithm}
\label{sec:newmark_algo}


\begin{algorithm}
  % \setstretch{1.30} % For article class
  \setSpacing{1.30}
  Initial conditions, $\bm x_0, \dot{\bm x}_0$. \linebreak
  $A_1 = (1 - \gamma) h, \quad
  B_1 = (\frac{1}{2} - \beta) h^2, \quad
  A_2 = \frac{1}{\beta h^2}, \quad
  B_2 = \frac{\gamma}{\beta h}$  \linebreak
  $\bm S_{lin} = \bm K + \frac{\gamma}{\beta h} \bm C + \frac{\beta}{h^2} \bm M$  \linebreak
  $\ddot{\bm x}_0 = \bm M^{-1}(\bm p_0 - \bm C \dot{\bm x}_0 - \bm K \bm x_0 -
  \bm f_{nl}\left( \bm u_0,\dot{\bm u}_0 \right))$  \linebreak
  \If{sensitivity}{
  \nonl$\bm V = [\bm I,\bm 0; \bm 0, \bm 0], \dot{\bm V} = [\bm 0,\bm
  0;\bm 0,\bm I]$ \\
  \nonl$\ddot{\bm V} = - \bm M^{-1} \left((\bm C + \bm f_{nl,\dot{\bm x}}) \dot{\bm
      V} + (\bm K  + \bm f_{nl,\bm x}) {\bm V} \right)$
}
\For{$n \gets 0$ \KwTo $nt$}{

  Prediction step (time integration): \linebreak
  $\dot{\bm x}_{n+1} = \dot{\bm x}_{n} + A_1 \ddot{\bm x}_n$ \linebreak
  $\bm x_{n+1} = {\bm x}_{n} + h \dot{\bm x}_n + B_1 \ddot{\bm x}_n$ \linebreak
  $\ddot{\bm x}_{n+1} = 0$

  Residual calculation: \linebreak
  $\bm r = \bm M \ddot{\bm x}_{n+1} + \bm f_{nl} + \bm f_{l} -  \bm p_{n+1}$

  \While{norm($\bm r$) > tol}{
    \nonl NR iteration (increment correction) \linebreak
    $\bm S_{eff} = \bm f_{nl, \bm x} + B_2\bm f_{nl, \dot{ \bm x}} + \bm S_{lin}$  \linebreak
    $\Delta \bm x = - \bm S_{eff}^{-1} \bm r$ \linebreak
    $ \ddot{\bm x}_{n+1} += A_2 \Delta \bm x$ \linebreak
    $ \dot{\bm x}_{n+1} += B_2 \Delta \bm x$ \linebreak
    $\bm x_{n+1} += \Delta \bm x$

    $\bm r = \bm M \ddot{\bm x}_{n+1} + \bm f_{nl} + \bm f_{l} -  \bm p_{n+1}$
  }
  \If{sensitivity}{
    prediction \linebreak
    $\bm V = \bm V + h\dot{\bm V} + B_1 \ddot{\bm V}$\linebreak
    $\dot{\bm V} = \dot{\bm V} + A_1 \ddot{\bm V}$\linebreak
    correction \linebreak
    $\bm S = \bm S_{lin} + \bm f_{nl, \bm x} + B_2 \bm f_{nl,
      \dot{\bm x}}$\linebreak
    $\bm S = \beta h^2 \bm S$\linebreak
    $\ddot{\bm V} = -\bm S^{-1}\left[
      (\bm C + \bm f_{nl, \dot{\bm x}} )\dot{\bm V} +
      (\bm K + \bm f_{nl, \bm x} ) \bm V  \right]$ \linebreak
    $\bm V = \bm V + \beta h^2 \ddot{\bm V}$ \linebreak
    $\dot{\bm V} = \dot{\bm V} + \gamma h \ddot{\bm V}$\linebreak
  }
}
  \caption{Nonlinear Newmark algorithm}
  \label{algo:nonlin_newmark}
\end{algorithm}

\subsection{Summary}

The Newmark-beta method used for implicit time integration was developed in
1959. Today it is one of the most widely used solvers for dynamics systems, and
for most problems it performs reasonable well. For problems with very stiff and
very flexible parts, i.e. where eigenvalues are very separated, it performs
poorly due to lack of high frequency numerical damping. Interested readers are
referred to \cite{bathe2012a}, which also suggest one alternative (it should be
noted that what Bathe calls the Newmark trapezoidal rule is mostly known as the
Newmark average acceleration method).


%%% Local Variables:
%%% mode: latex
%%% TeX-master: "../../report"
%%% End:
