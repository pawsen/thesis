
\chapter{From identification to design}
\label{chap:ident_to_design}

This chapter deals with the prediction of behaviour of nonlinear system, after
nonlinear components have been determined. Paraphrased it could be called
\textit{virtual prototyping}. Here methods to detect bifurcations, stability and
internal resonance are presented. The long-term ambition is to use this
knowledge to understand and improve design, taking nonlinear behaviour into
account.


This chapter represent a change of methodology. Where the methods for
identification in the previous chapter relied exclusively on time signals, the
methods of this chapter relies exclusively on FE models.
The information from identification is used to build
an accurate computer model.
By accurate model is meant a model where the $n$ lowest modes match
the experimental ones.


Two methods are treated: Harmonic balance(HB) continuation, used for computation
of NFRC and detecting and identifying bifurcations, and nonlinear normal
modes(NNM) continuation, used to detect internal resonance.

The concept of NNMs is reviewed in appendix \ref{sec:nonl-norm-modes}. In the
following it is enough to know that nonlinear normal modes is considered an
extension to linear normal modes(LNM), but without the mathematical properties
of LNMs. Where LNMs decouple eoms, ie. a solution can be written as a linear
combinations of LNMs and they are invariant, this is not the case for NNMs.

Instead NNMs are defined as \textit{(non-necessarily synchronous) periodic
  motion of the system}\autocite{kerschen2009a} and computed for the undamped
and unforced system, the hamiltonian system.
Synchronous means that all parts of a system reaches their extreme values and
passes through zero at the same time, ie. vibration in unison.
In the case of internal resonance, where one part vibrates at a integer number
of another parts frequency, the vibration can still be periodic but it is no
longer in unison. Thus NNMs are simply periodic oscillations of a conservative
system.

NNM have the following properties/features
\begin{itemize}
\item \textit{Frequency-energy dependence}:\\
  Nonlinear systems shows a frequency-energy dependence of their oscillations,
  eg. through hardening or softening behaviour. NNMs plotted in a
  frequency-energy plot show this dependency.
\item \textit{Internal resonances}:\\
  NNMs may interacts and exchange energy at incommensurate frequencies
\item \textit{Traces the lotus of NFRCs}:\\
  For structures with low damping, NNMs traces the lotus of resonance peaks.
\end{itemize}




% Måske til opsummering:
% (This is a step forward from the traditional methods used today where designs are
% based on shaping resonance(by solving eigenvalue problems) and validated with
% modal responses from experimental data.)


%%% Local Variables:
%%% mode: latex
%%% TeX-master: "../report"
%%% End:
