
\chapter{From identification to design}
\label{chap:ident_to_design}

This chapter deals with the prediction of behaviour of nonlinear system, after
nonlinear components have been determined. Paraphrased it could be called
\textit{virtual prototyping}. Here methods to detect bifurcations, stability and
internal resonance are presented. The long-term ambition is to use this
knowledge to understand and improve design, taking nonlinear behaviour into
account.


This chapter represent a change of methodology. Where the methods for
identification in the previous chapter relied exclusively on time signals, the
methods of this chapter relies exclusively on FE models.
The information from identification is used to build
an accurate computer model.

Two methods are treated: Harmonic balance(HB) continuation, used for computation
of NFRC and detecting and identifying bifurcations, and nonlinear normal
modes(NNM) continuation, used to detect internal resonance.

Before presenting the methods, the concept of NNMs is reviewed.



% Måske til opsummering:
% (This is a step forward from the traditional methods used today where designs are
% based on shaping resonance(by solving eigenvalue problems) and validated with
% modal responses from experimental data.)


%%% Local Variables:
%%% mode: latex
%%% TeX-master: "../report"
%%% End:
