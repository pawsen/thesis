
\chapter{Methods for dection, characterization and estimation}
\label{chap:methods_dec_char_est}

This chapter deals with the identification process as depicted in figure
\ref{fig:ident_process}, and repeated below, treating localised nonlinearities.

\begin{figure}[ht!]
  \centering
  \begin{mdframed}
    \begin{enumerate}
    \item Detection: {\textit Is there?}\\
      Ascertain if nonlinearity exist in the structural behavior, e.g. yes or
      no.
    \item Characterisation: {\textit Where, what and how?}
      \begin{itemize}
      \item Localize the nonlinearity, e.g. at the joint
      \item Determine the type of nonlinearity e.g.  Coulomb friction\\
        More general: is it stiffness or damping nonlinearity or both. In the
        case of stiffness: is it hardening or softening
      \item Select the functional form of the nonlinearity, e.g.
        $f(x,\dot x) = c \sign (\dot y)$
      \end{itemize}
    \item Parameter estimation: {\textit How much?}\\
      Calculate the coefficients of the nonlinearity model, e.g. $c = 5.47$.\\
      (Ideally the uncertainty should be quantified, e.g. in a probabilistic
      sense, $c \sim N(5.47,1)$. But that is a very difficult task and not
      within the scope of this thesis)
    \end{enumerate}
  \end{mdframed}
  \caption{Identification process for nonlinear structural models}
\end{figure}


Methods for detection are only referred, as detection is usual deemed easy
within the framework of localised nonlinearities \autocite{noel2016a}.
Characterisation is done by wavelet transform and restoring force surface, while
estimation is done by FNSI.


%%% Local Variables:
%%% mode: latex
%%% TeX-master: "../report"
%%% End:
