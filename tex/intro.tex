\chapter{Introduction}
\label{sec:introduction}

In all structures, nonlinearities which will affect the dynamic behavior of the
structure are present to some extend. Depending on excitation level, the
structure will exhibit linear or nonlinear dynamics. Nonlinearities have always
existed, but are often neglected.

To get a further understanding of the nonlinear effects present, an efficient
set of tools for identification, characterization and estimation of
nonlinearities in engineering structures from experimental observations is be
useful.

The motivation for this thesis is to provide that understanding, hopefully
giving the reader a ``toolbox'' applicable to nonlinear systems. A toolbox is to
be understood as a collection of methods, applicable to nonlinear problems.
Like modal testing is one method in the linear toolbox.


To exemplify and test this toolbox, a part of the thesis is dedicated to
developing, implementing and exemplifying the methods of the toolbox
numerically.

By using such a ``numerical toolbox'', further understanding of the nonlinear
dynamics can be obtained by simulation, then what is gained by pure experiment,
and hopefully assist in virtual prototyping.

One requirement for the toolbox is that it works on experimental data, e.g. time
series, alone. Methods exists which can do identification and estimation, but
they often require either that the Equations of Motions(EOM) are assembled or a
detailed Finite Element Model(FEM) is constructed, and are thus often difficult
and time consuming to use.

Another aspect of the toolbox is to quantify the size, or importance, of
nonlinearity:

Even if the area of nonlinear identification and modeling have received a great
deal more attention within the last ten years, it is still far behind the linear
counterpart, both in theory and application. Thus a nonlinear toolbox certainly
requires some specialization to use.
If the nonlinearity is weak it might suffice with linear analysis, giving access
to all the traditional methods, with reduced time spend on the analysis as a
consequence.

% identification, characterization and modeling of localized nonlinearities in
% structural dynamics. The goal of creating such a toolbox is to get 

% Such a toolbox already exists to some degree, e.g. Ni2D \citep{Ni2D}, developed
% at Liège University.

% Even if the area of nonlinear identification and modeling have received a great
% deal more attention within the last ten years, it is still far behind the linear
% counterpart, both in theory and application. Thus a nonlinear toolbox
% usefulness is still somewhat limited and certainly requires some specialization
% to use. This is also true for the Ni2D software.

% The 
% The identification is done on experimental time data and 


\subsection{Why nonlinear modeling}
\label{sec:why-nonl-model}

For nonlinear systems, the superposition and thus invariance of modes and
uniqueness of solutions (e.g. the forced steady state response is dependent on
the initial transient behavior) does no longer hold, and many of the techniques
from linear analysis cannot be used.

Linear system is an exception. If excited hard enough, all system displays some
nonlinear behavior. But often the nonlinearity stems from joints (damping),
contact (stiffness) or geometrical nonlinearities, which is why most of the
literature today treats localized nonlinearity, assuming the location of the
nonlinearity is known.
Another reason for dealing with localized nonlinearities, is that no robust
method for localization exists. In his thesis \textcite{kragh2010a} test different
methods for localizing nonlinearity and concludes that ``it was not possible to
obtain consistent localization of the nonlinearities'' even for simple structures.

With the introduction of ever lighter structures, exotic materials, high speed
machinery, etc., nonlinear tools are needed to fully understand the dynamics.
Also to determine if nonlinear analysis is indeed needed, since this kind of
analysis requires substantial more effort than linear analysis would do.

For a general introduction to nonlinear dynamics, the textbook by
\textcite{juel2003a} can be used.



\subsection{Nonlinear system identification?}
\label{sec:nonl-syst-ident}

\textcite{kerschen2006a} proposed to regard the identification of nonlinear
structural models as a progression through three steps: {\textit detection}, {\textit
  characterization} and {\textit estimation}, as outlined in figure
\ref{fig:ident_process}

The first book on the topic was \textcite{worden2000nonlinearity}, and even
though many new methods has been introduced since then, it still gives a good
introduction to the subject as well as a overview of the common types of
nonlinearity.

A comprehensive review of the development in nonlinear system identification was
given by \parencite{kerschen2006a} and just recently by \textcite{noel2016a}. For
comparison of the many techniques in use, the reader is refereed to these
reviews. In this thesis, the choice of technique will be motivated but
alternative techniques will not necessarily be mentioned.


%\begin{infobox}]

\begin{figure}[ht!]
  \centering
  \begin{mdframed}
    \begin{enumerate}
    \item Detection: {\textit Is there?}\\
      Ascertain if nonlinearity exist in the structural behavior, e.g. yes or
      no.
    \item Characterisation: {\textit Where, what and how?}
      \begin{itemize}
      \item Localize the nonlinearity, e.g. at the joint
      \item Determine the type of nonlinearity e.g.  Coulomb friction\\
        More general: is it stiffness or damping nonlinearity or both. In the
        case of stiffness: is it hardening or softening
      \item Select the functional form of the nonlinearity, e.g.
        $f(x,\dot x) = c \sign (\dot y)$
      \end{itemize}
    \item Parameter estimation: {\textit How much?}\\
      Calculate the coefficients of the nonlinearity model, e.g. $c = 5.47$.\\
      (Ideally the uncertainty should be quantified, e.g. in a probabilistic
      sense, $c \sim N(5.47,1)$. But that is a very difficult task and not
      within the scope of this thesis)
    \end{enumerate}
  \end{mdframed}
  \caption{Identification process for nonlinear structural models}
  \label{fig:ident_process}
\end{figure}

% \end{infobox}


\subsubsection{Detection}
\label{sec:detection}

Of the three steps, detection is the easiest. During test, the structure should
be excited by a sine-sweep and a mere visual inspection of the time series will
show if nonlinearity is present. Signs includes skewness of the envelope,
discontinuity, jumps and lack of invariance with increasing force level. The
excitation level needs to be at an amplitude where the nonlinearity is
activated.

Random excitation is in general not useful, as the randomness of the amplitude
and phase of the excitation creates ``linearized'' frequency response functions
(FRFs). At least multiple test with different rms levels are required, and still
then it might be difficult to excite the nonlinearities, since the total power
of the input spectrum is spread over the band-limited frequency range used.

The use of impact excitation, as often used in linear analysis, suffers from the
same problems as random excitation. That is, the input is a broad spectrum and
the energy associated with each frequency is low.


Formal methods for detection includes

\begin{itemize}
\item Homogeneity check \\
  Comparing the response of two sweeps with different forcing and calculating
  the cross correlation. It is a test of superposition, by testing if the two FRFs
  normalized with forcing overlay as they will for linear systems.
\item (ordinary) Coherence function \\
  The coherence function compares power spectral densities (PSDs) and are
  required to be unity for all accessible frequencies for the system to be
  linear {\textit and} free of noise. The advantage is that only one test is needed,
  but the method does not distinguish between cases of noise and
  nonlinearity. Instead it is recommend do use:
\item Hilbert transforms \\
  This method detects nonlinearity by doing a Hilbert transform of the FRF,
  which is invariant for a linear FRF.
  A Hilbert transform also only requires one data set and is more sensitive to
  nonlinearity than the coherence function, but still reasonable easy to
  implement. \textcite{kragh2010a} shows that the homogeneity check is superior to
  the Hilbert transforms, having higher sensitivity to nonlinearity.
% \item Wavelet transforms \\
%   Maps a time-history to a time-frequency representation. Fourier transform
%   cannot be used, since it is a one-to-one transformation from time to frequency
%   domain.
\end{itemize}

For all of these methods it is a requirement that the nonlinearities are
activated, e.g. the forcing level end frequency interval should be chosen
adequately. Also, the methods are better at detecting nonlinear stiffness than
nonlinear damping. This is due to the fact that the resonance peak is not
shifted as with the stiffness nonlinearity case. Since the FRF is not shifted
but only lowered, the cross correlation coefficient will not decrease as much as
in the stiffness nonlinearity case. In general damping nonlinearity is difficult
to identify and will only be touched briefly in this thesis.

\subsubsection{Characterisation}
\label{sec:characterization}

The second step is the most important and also the most difficult, when
localization is not considered.
This step seeks to identify the aspects of the motion that drives the
nonlinearity, e.g. displacement or velocity and a representative functional form
to represent the nonlinearity.


The most used technique is the {\textit restoring force surface} (RFS). The RFS
provides information of restoring force in the excited range. To visualize the
functional of the restoring force and the dissipative force, two slices in the
RFS is made: at zero displacement and zero velocity. The functional form is then
found by simple inspection of the slices or fitting polynomials and perform
goodness of fit.

Another characterisation method, the Morlet wavelet transform, is used to
visualise how the frequency content changes with amplitude, a consequence of the
energy-frequency dependency for nonlinear vibrations. The visualised
instantaneous spectrum can both be used for detection of nonlinearity and help
estimating the type of nonlinearity,

% Other characterization methods include blackbox modeling, which do
% characterization without regards to the underlying physics, instead using a rich
% and flexible mathematical structure to capture all relevant dynamics.


\subsubsection{Estimation}
\label{sec:estimation}

The RFS method can be used for estimation as well, fitting the functional form
to the surface. But in order to scale the RFS correct, an estimate of the mass
(or inertia) is needed or the full EOM has to be assembled. This is often
difficult for MDOF systems and violates the ambition for the toolbox: that it
works on time series exclusively.

A newer method, introduced by \textcite{noel2013a}, is the frequency-domain
nonlinear subspace identification(FNSI). This method works on
time series alone and is able to estimate nonlinear parameters and the
underlaying linear transfer function.

% Frequency domain
% methods should generally be less sensitive to noise than time domain
% methods. {\bf men den har jeg ikke skimmet endnu. Det skulle være en lovende
%   metode}


% the parameter estimation is found  The technique requires the measurement devices (accelerometers) to be
% places as close to the nonlinearities as possible.  As already stated, the RFS
% can only estimate parameters within the excited regime.



\subsection{Beyond nonlinear system identification}
\label{sec:beyond-nonl-syst}

When the identification steps is completed, a structural model can be build from
a FEM of the underlying linear structure with the identified nonlinearities
incorporated. It shall be thought of as (larger) chunks of linear sections
connected through nonlinear elements. To reduce the computational time, the
linear model is often reduced using the Craig-Bampton reduction technique
\autocite{craig1968a}.

If the predictions from the nonlinear FEM can be verified by the
experimental results, the numerical model can be used to {\textit get further
  understanding of the nonlinear dynamics}. The latter is the whole goal of the
identification, as it allows for uncovering new nonlinear regimes of motion and
to make design modifications. The concept of using numerical experiments to
assist with the design is referred to as {\textit virtual prototyping}.


\subsubsection{Internal resonance}
\label{sec:internal-resonance}


Nonlinear resonances are investigated using an extension of linear normal modes
(LNMs) to nonlinear theory, the nonlinear normal modes (NNMs) described in
appendix \ref{sec:nonl-norm-modes}.
Where a LNM is interpreted as the deformation along the axis of the vibrating
structure or the rotation, a NNM does not have such a clear interpretation.
An NNM is said to be a periodic oscillation of the underlaying undamped and
unforced nonlinear system and depends on the frequency and energy of the
system.


\subsubsection{Bifurcation}
\label{sec:bifurcation}


Using the HB method, nonlinear forced response curves(NFRCs) for a periodic
excitation are calculated. The transition from a stable periodic solution to an
unstable solution occurs through bifurcations. The type of bifurcation is used
to qualify the type of unstable solution emerging.
% , following the outline in
% \textcite{detroux2015a}.


% \subsection{Thesis outline}
% \label{sec:thesis-outline}

% Section 2 introduces the theoretical methods used: NNMs, RFS and FNSI,

% Section 3 introduces the numerical methods: FEM discretization and model
% reduction, Newmark time integration, harmonic balance and continuation for
% calculating NNMs and NFRF.

% It also briefly discusses methods for integrating and differentiating time
% signals and filtering techniques.

% Section 4 introduce identification and simulation of benchmark data from a
% nonlinear system, the COST beam, \cite{GOLINVAL2003}, which have a cubic
% stiffening nonlinear due to geometry and a squared nonlinearity due to clamping.

% Section 5 introduces numerical experiments to investigate how the methods
% performs and their sensitivity to noise.

% Finally section 6 contains an discussion and conclusion and suggest further
% studies and implementations.

% There will not be given much attention to the detection of nonlinearity.

% \subsection{Summary}
% \label{sec:intro-summary}


% With a toolbox for detection, characterization and estimation of nonlinear
% parameters, numerical modeling will provide additional insight into the dynamics
% of the system.

%%% Local Variables:
%%% mode: latex
%%% TeX-master: "../report"
%%% End:
