% -*- coding: utf-8 -*-

% \documentclass[a4paper, english,12pt]{article}
\documentclass[a4paper,12pt,twoside,openright]{memoir}
\usepackage[
    backend=biber,
    %style=authoryear-icomp,% numeric-comp, %
    style=numeric-comp,
    %style=numeric,
    sortlocale=none,%de_DE,
    autocite=superscript,
   % natbib=true,
    url=false,
    doi=true,
    eprint=false,
    sortcites   = true,
    citetracker
    ]{biblatex}
\addbibresource{biblio.bib}

\usepackage[rmargin=3cm,tmargin=3cm,bmargin=3cm]{geometry}		% Ændring af margener
\usepackage[utf8]{inputenc} 		% Skal passe til editorens indstillinger - specificerer input encoding
\usepackage[english]{babel}			% danske orddelinger
\usepackage[T1]{fontenc} 			% fonte (output)
\usepackage{lmodern} 				% vektor fonte
\usepackage{color}

% \setlength{\parindent}{0pt}			% Intet indryk i nye afsnit
\usepackage{mathtools} 				% matematik - understøtter muligheden for at bruge \eqref{}
%\mathtoolsset{showonlyrefs,showmanualtags} %Viser kun formelnumre der er refereret til!
\usepackage{amsmath,amssymb,bm} % Husk disse pakker når du laver matematik!

\usepackage[linesnumbered,ruled,vlined]{algorithm2e}
\let\oldnl\nl% Store \nl in \oldnl
% Remove line number for one line
\newcommand{\nonl}{\renewcommand{\nl}{\let\nl\oldnl}}

%multiple columns on a page
\usepackage{multicol}

% for figures
\usepackage[font={small,it},labelfont={sc, bf}]{caption}
\usepackage{graphicx, subcaption}
% for matplotlib2tikz
\usepackage{pgfplots}
\usepackage{pgfplotstable}
%\usepackage{fontspec} % XeTex eller LuaTeX krævet
\newlength\figureheight
\newlength\figurewidth
\pgfplotsset{compat=newest}
\usepgfplotslibrary{groupplots}

\usepackage{placeins}               % \FloatBarrier
\usepackage{tikz}
% \usetikzlibrary{external}
\usetikzlibrary{shapes,arrows,positioning}%,external}
%\tikzexternalize[prefix=build/]
\usepackage{tikzscale}
\usepackage[]{mdframed}
\usepackage{multirow} % for tables
\usepackage{chngcntr} % reset numbering after each section
\usepackage{import}
\usepackage{listings}

\usepackage[acronym,nomain,toc,nonumberlist]{glossaries} % ,xindy
\makeglossaries
\usepackage[]{hyperref}

% rækkefølge til billedformat
\DeclareGraphicsExtensions{.tikz,.pdf,.png,.jpg}
\graphicspath{{./fig/}} 			% stivej til bibliotek med figurer


\newcounter{infobox}[section]
\newenvironment{infobox}[1][]{%
    \refstepcounter{infobox}
    \begin{mdframed}[%
        frametitle={Infobox \theinfobox\ #1},
        skipabove=\baselineskip plus 2pt minus 1pt,
        skipbelow=\baselineskip plus 2pt minus 1pt,
        linewidth=0.5pt,
        frametitlerule=true,
        frametitlebackgroundcolor=gray!30
    ]%
}{%
    \end{mdframed}
}

% Bemærk at "nye" kommandoer bør stå efter \usepackage

% Nedenstående tre kommandoer ændrer figur-, tabel- og ligningsnummereringen
% således at den starter forfra for hvert afsnit og tilføjer sektionsnummeret.
\renewcommand{\theequation}{\thesection.\arabic{equation}}		% \arabic{section}
\renewcommand{\thefigure}{\thesection.\arabic{figure}}
\renewcommand{\thetable}{\thesection.\arabic{table}}
\counterwithin*{equation}{section} % reset numbering after each section
\counterwithin*{figure}{section} % reset numbering after each section
\counterwithin*{table}{section} % reset numbering after each section

% show line (symbol) in text. Commands:
% \sampleline{}
% \sampleline{dashed}
% \sampleline{dotted}
% \sampleline{dash pattern=on .7em off .2em on .2em off .2em}
% \sampleline{dash pattern=on .7em off .2em on .05em off .2em}
\DeclareRobustCommand\sampleline[1]{%
  \tikz\draw[#1] (0,0) (0,\the\dimexpr\fontdimen22\textfont2\relax)
  -- (2em,\the\dimexpr\fontdimen22\textfont2\relax);%
}

\tikzset{%
  do path picture/.style={%
    path picture={%
      \pgfpointdiff{\pgfpointanchor{path picture bounding box}{south west}}%
      {\pgfpointanchor{path picture bounding box}{north east}}%
      \pgfgetlastxy\x\y%
      \tikzset{x=\x/2,y=\y/2}%
      #1
    }
  },
  sin wave/.style={do path picture={
      \draw [line cap=round] (-3/4,0)
      sin (-3/8,1/2) cos (0,0) sin (3/8,-1/2) cos (3/4,0);
    }},
  cross/.style={do path picture={
      \draw [line cap=round] (-1,-1) -- (1,1) (-1,1) -- (1,-1);
    }},
  plus/.style={do path picture={
      \draw [line cap=round] (-3/4,0) -- (3/4,0) (0,-3/4) -- (0,3/4);
    }}
}

% Lav cirkel uden om bogstav. Kræver pakken \usepackage{tikz}. Benyttes med fx. \mycirc{A}
\newcommand*\mycirc[1]{
  \tikz[baseline=(C.base)]\node[draw,circle,inner sep=0.5pt](C) {#1};\! }

% Laver Lapace L ved \L.
\renewcommand{\L}[1]{\ensuremath{\mathcal{L} \{ {#1} \} }}
%\renewcommand{\*}{\cdot}
\newcommand{\p}{\ensuremath{\partial}}
\renewcommand{\d}[1]{\ensuremath{\operatorname{d}\!{#1}}}
\renewcommand{\O}{\ensuremath{\mathcal{O}}}
\renewcommand{\d}[1]{\ensuremath{\operatorname{d}\!{#1}}}
\renewcommand{\d}{\ensuremath{\operatorname{d}\!}}
% \renewcommand{\(}{\ensuremath{\left( }}
% \renewcommand{\)}{\ensuremath{\right) }}
\renewcommand{\Re}{\textup{Re}}
\renewcommand{\Im}{\textup{Im}}
\newcommand{\degreeC}{\ensuremath{^\circ \mathrm{C}}}
\newcommand{\norm}[1]{\left\lVert #1 \right\rVert}

\DeclareMathOperator{\sign}{sign}

\newcommand{\Arch}{\operatorname{\mathit{A\kern-.06em r}}} % http://wikipedia.org/wiki/Archimedes-Zahl
\newcommand{\Biot}{\operatorname{\mathit{B\kern-.06em i}}} % http://wikipedia.org/wiki/Biot-Zahl
\newcommand{\Cauc}{\operatorname{\mathit{C\kern-.07em a}}} % http://wikipedia.org/wiki/Cauchy-Zahl
\newcommand{\Damk}{\operatorname{\mathit{D\kern-.06em a}}} % http://wikipedia.org/wiki/Damk%C3%B6hler-Zahl
\newcommand{\Eule}{\operatorname{\mathit{E\kern-.03em u}}} % http://wikipedia.org/wiki/Euler-Zahl
\newcommand{\Four}{\operatorname{\mathit{F\kern-.10em o}}} % http://wikipedia.org/wiki/Fourier-Zahl
\newcommand{\Frou}{\operatorname{\mathit{F\kern-.07em r}}} % http://wikipedia.org/wiki/Froude-Zahl
\newcommand{\Gras}{\operatorname{\mathit{G\kern-.05em r}}} % http://wikipedia.org/wiki/Grashof-Zahl
\newcommand{\Karl}{\operatorname{\mathit{K\kern-.11em a}}} % http://wikipedia.org/wiki/Karlovitz-Zahl
\newcommand{\Knud}{\operatorname{\mathit{K\kern-.11em n}}} % http://wikipedia.org/wiki/Knudsen-Zahl
\newcommand{\Lewi}{\operatorname{\mathit{L\kern-.05em e}}} % http://wikipedia.org/wiki/Lewis-Zahl
\newcommand{\Mach}{\operatorname{\mathit{M\kern-.10em a}}} % http://wikipedia.org/wiki/Mach-Zahl
\newcommand{\Nuss}{\operatorname{\mathit{N\kern-.09em u}}} % http://wikipedia.org/wiki/Nusselt-Zahl
\newcommand{\Pecl}{\operatorname{\mathit{P\kern-.08em e}}} % http://wikipedia.org/wiki/P%C3%A9clet-Zahl
\newcommand{\Pran}{\operatorname{\mathit{P\kern-.03em r}}} % http://wikipedia.org/wiki/Prandtl-Zahl
\newcommand{\Rayl}{\operatorname{\mathit{R\kern-.04em a}}} % http://wikipedia.org/wiki/Rayleigh-Zahl
\newcommand{\Reyn}{\operatorname{\mathit{R\kern-.04em e}}} % http://wikipedia.org/wiki/Reynolds-Zahl
\newcommand{\Schm}{\operatorname{\mathit{S\kern-.07em c}}} % http://wikipedia.org/wiki/Schmidt-Zahl
\newcommand{\Sher}{\operatorname{\mathit{S\kern-.07em h}}} % http://wikipedia.org/wiki/Sherwood-Zahl
\newcommand{\Stro}{\operatorname{\mathit{S\kern-.07em r}}} % http://wikipedia.org/wiki/Strouhal-Zahl
\newcommand{\Webe}{\operatorname{\mathit{W\kern-.14em e}}} % http://wikipedia.org/wiki/Weber-Zahl


%%% Local Variables:
%%% mode: latex
%%% TeX-master: "report"
%%% End:
